% Options for packages loaded elsewhere
\PassOptionsToPackage{unicode}{hyperref}
\PassOptionsToPackage{hyphens}{url}
\documentclass[
]{article}
\usepackage{xcolor}
\usepackage[margin=1in]{geometry}
\usepackage{amsmath,amssymb}
\setcounter{secnumdepth}{-\maxdimen} % remove section numbering
\usepackage{iftex}
\ifPDFTeX
  \usepackage[T1]{fontenc}
  \usepackage[utf8]{inputenc}
  \usepackage{textcomp} % provide euro and other symbols
\else % if luatex or xetex
  \usepackage{unicode-math} % this also loads fontspec
  \defaultfontfeatures{Scale=MatchLowercase}
  \defaultfontfeatures[\rmfamily]{Ligatures=TeX,Scale=1}
\fi
\usepackage{lmodern}
\ifPDFTeX\else
  % xetex/luatex font selection
\fi
% Use upquote if available, for straight quotes in verbatim environments
\IfFileExists{upquote.sty}{\usepackage{upquote}}{}
\IfFileExists{microtype.sty}{% use microtype if available
  \usepackage[]{microtype}
  \UseMicrotypeSet[protrusion]{basicmath} % disable protrusion for tt fonts
}{}
\makeatletter
\@ifundefined{KOMAClassName}{% if non-KOMA class
  \IfFileExists{parskip.sty}{%
    \usepackage{parskip}
  }{% else
    \setlength{\parindent}{0pt}
    \setlength{\parskip}{6pt plus 2pt minus 1pt}}
}{% if KOMA class
  \KOMAoptions{parskip=half}}
\makeatother
\usepackage{graphicx}
\makeatletter
\newsavebox\pandoc@box
\newcommand*\pandocbounded[1]{% scales image to fit in text height/width
  \sbox\pandoc@box{#1}%
  \Gscale@div\@tempa{\textheight}{\dimexpr\ht\pandoc@box+\dp\pandoc@box\relax}%
  \Gscale@div\@tempb{\linewidth}{\wd\pandoc@box}%
  \ifdim\@tempb\p@<\@tempa\p@\let\@tempa\@tempb\fi% select the smaller of both
  \ifdim\@tempa\p@<\p@\scalebox{\@tempa}{\usebox\pandoc@box}%
  \else\usebox{\pandoc@box}%
  \fi%
}
% Set default figure placement to htbp
\def\fps@figure{htbp}
\makeatother
% definitions for citeproc citations
\NewDocumentCommand\citeproctext{}{}
\NewDocumentCommand\citeproc{mm}{%
  \begingroup\def\citeproctext{#2}\cite{#1}\endgroup}
\makeatletter
 % allow citations to break across lines
 \let\@cite@ofmt\@firstofone
 % avoid brackets around text for \cite:
 \def\@biblabel#1{}
 \def\@cite#1#2{{#1\if@tempswa , #2\fi}}
\makeatother
\newlength{\cslhangindent}
\setlength{\cslhangindent}{1.5em}
\newlength{\csllabelwidth}
\setlength{\csllabelwidth}{3em}
\newenvironment{CSLReferences}[2] % #1 hanging-indent, #2 entry-spacing
 {\begin{list}{}{%
  \setlength{\itemindent}{0pt}
  \setlength{\leftmargin}{0pt}
  \setlength{\parsep}{0pt}
  % turn on hanging indent if param 1 is 1
  \ifodd #1
   \setlength{\leftmargin}{\cslhangindent}
   \setlength{\itemindent}{-1\cslhangindent}
  \fi
  % set entry spacing
  \setlength{\itemsep}{#2\baselineskip}}}
 {\end{list}}
\usepackage{calc}
\newcommand{\CSLBlock}[1]{\hfill\break\parbox[t]{\linewidth}{\strut\ignorespaces#1\strut}}
\newcommand{\CSLLeftMargin}[1]{\parbox[t]{\csllabelwidth}{\strut#1\strut}}
\newcommand{\CSLRightInline}[1]{\parbox[t]{\linewidth - \csllabelwidth}{\strut#1\strut}}
\newcommand{\CSLIndent}[1]{\hspace{\cslhangindent}#1}
\setlength{\emergencystretch}{3em} % prevent overfull lines
\providecommand{\tightlist}{%
  \setlength{\itemsep}{0pt}\setlength{\parskip}{0pt}}
\usepackage{bookmark}
\IfFileExists{xurl.sty}{\usepackage{xurl}}{} % add URL line breaks if available
\urlstyle{same}
\hypersetup{
  pdftitle={Topic and Workflow Summary},
  pdfauthor={Lauren Berger},
  hidelinks,
  pdfcreator={LaTeX via pandoc}}

\title{Topic and Workflow Summary}
\author{Lauren Berger}
\date{}

\begin{document}
\maketitle

\textbf{Historical contingency in post-fire plant communities: analyzing
diversity in managed forests}

\textbf{Author:} Lauren Berger

\subsection{Literature review (\textasciitilde5
paragraphs)}\label{literature-review-5-paragraphs}

\subsubsection{Paragraph 1 --- Problem \& significance (big‑picture
reviews)}\label{paragraph-1-problem-significance-bigpicture-reviews}

Wildfires in the western U.S. have become more frequent and
severe(Halofsky et al., 2020). High-severity fires (\textgreater95\%
tree mortality) have increased eight-fold due to long-term fire
exclusion and climate changeParks \& Abatzoglou (2020). Meanwhile, many
forests in this region are managed for timber production. As a result,
the compounded disturbances of high-severity fire and logging
increasingly co-occur, potentially diminishing biodiversity beyond
either event alone. While mixed-severity fires create habitat mosaics
that support biodiversity,(Ponisio, 2020a) high-severity fires often
simplify communities by reducing local diversity and species
turnover(Steel et al., 2022). However, few studies have examined the
influence of historical contingency in driving these patterns. With
wildfire severity and prevalence projected to rise,(Parks \& Abatzoglou,
2020) understanding how fire shapes community assembly in managed
forests is crucial for predicting and managing biodiversity at local and
regional scales. \ldots{}

\subsubsection{Paragraph 2--3 --- Past work \& data landscape
(who/where/how;
gaps)}\label{paragraph-23-past-work-data-landscape-whowherehow-gaps}

Historical contingency provides a framework to interpret the effects of
varying disturbance legacies on community assembly, leading to
measurable differences between sites under similar conditions(Fukami,
2015). Two crucial components of historical contingency ---site legacies
(enduring characteristics from past events)(Johnstone et al., 2016) and
priority effects (sequence and timing of species arrivals)(Fukami,
2015)---are particularly relevant when assessing post-fire ecosystems.

As plant-pollinator communities are both fire-sensitive and central to
community assembly, they offer an ideal model for evaluating these
concepts(Ponisio, 2020b). The only two studies on site legacies in
harvested forests found that past stand age influences plant
composition. Older stands tend to regenerate shrub-dominated assemblages
resembling pre-fire communities, while younger stands often contain more
disturbance-adapted vegetationBowd et al. (2021). Yet, effects on local
species (α diversity) and turnover (β diversity) remain under-examined,
and no studies have examined the implications for pollinator
communities. Furthermore, research on plant--pollinator response to fire
rarely considers the temporal order of community assembly. As
high-severity fire eliminates both plant and bee communities, input from
nearby refugia is necessary for re-colonization.Ulyshen et al. (2024)
Given the increasing extent of high severity fire, priority effects may
intensify filtering, favoring generalist bees and plants capable of
long-distance dispersal. While these early-arriving species can rapidly
establish populations, they often overlap in ecological roles, leading
to communities with higher functional redundancy but lower diversity.
\ldots{}

\subsubsection{Paragraph 4 --- Purpose \& why
now}\label{paragraph-4-purpose-why-now}

This project will advance knowledge of how disturbance and historical
contingency interact to shape ecological communities, addressing a
central challenge in disturbance ecology. This research extends
community assembly theory to a real-world setting of overlapping
disturbances, measuring how site legacies and priority effects influence
biodiversity outcomes. Methodologically, this work integrates remote
sensing with fine-scale field collected ecological network data and
analysis to quantify how fire extent alters diversity, redundancy, and
turnover. This combination of landscape-scale disturbance mapping with
species interaction networks represents a novel approach that can be
applied broadly to other ecosystems experiencing compounded
disturbances, such as hurricanes, floods, and land-use change.

In the Pacific Northwest, both private and public forest managers are
increasingly motivated to conserve pollinator and native plant
species(Rivers et al., 2018). This urgency stems in part from the
pending ESA listing of several regional bee species, which could prompt
new regulatory measures. Pollinators also provide essential ecosystem
services, supporting an estimated two-thirds of global food
production(Khalifa et al., 2021). Forestry practices that sustain
healthy plant--pollinator communities are not only important for
biodiversity conservation but are also critical for ensuring economic
security. \ldots{}

\subsubsection{Paragraph 5 --- Hypotheses (directional) \& brief
rationale}\label{paragraph-5-hypotheses-directional-brief-rationale}

Due to the influence of historical contingency, I predict that variation
in biological legacies associated with pre-fire stand age and subsequent
species filtering via priority effects will generate measurable
differences in post-fire plant and pollinator communities. Specifically,
I predict that sites embedded within increasingly high-severity patch
areas will exhibit: (H1) lower taxonomic plant and bee α diversity and
(H2) lower β-diversity. In all hypotheses, I expect stand age pre-fire
to moderate these effects: older stands retain biological legacies that
buffer diversity loss, while younger stands display exaggerated
responses.

\subsection{Dataset identification}\label{dataset-identification}

\begin{itemize}
\tightlist
\item
  Holiday Farm Fire Veg.csv, FireSeverity.csv
\end{itemize}

\subsubsection{Workflow plan (prose)}\label{workflow-plan-prose}

\subsubsection{1) Cleaning \& validation:}\label{cleaning-validation}

All vegetation data will be cleaned utilizing prep1.rmd to identify NA's
in data and compare plant names and spelling to TNRS database.

To confirm that species ID is appropriate for the Holiday Farm Fire
area, data will be exported from OregonFlora utilzing a polygon that
encompasses the sampling perimeters and surrounding area, and given a
likeliness ranking based on species recording, utilizing the quantile
function in R. All species given a rank of ``Unlikey'' or ``NA'' will be
checked in OregonFlora for verification that recording is highly
unlikely, and and dropped from analysis.

To determine outliers and possible miscounts in vegetation data, I will
apply \ldots{} add in what I've been doing in sheet.

Units are already standardized across datasets

\subsubsection{2)
Aggregations/derivations:}\label{aggregationsderivations}

I will take the steps below to aggregate the data to filter for year and
fire

Filter for only Holiday Farm Fire Stands, removing sites containg 6 and
all 800 sites since these were survey a limited number of times

veg\_HFF \textless- veg \%\textgreater\% filter( str\_detect(Stand,
regex(``holiday'', ignore\_case = TRUE)), Site != ``6'',
!str\_detect(Stand, ``800'') )

Again, I will filter the data so only the year 2025 is being viewed

veg\_2025 \textless- veg\_HFF \%\textgreater\% mutate(Date = ymd(Date))
\%\textgreater\%\\
filter(year(Date) == 2025)

I will include the code below to summarize species abundance per each
Stand

Get unique site and species names stands \textless-
unique(veg\_2025\(Stand)
species <- unique(veg_2025\)PlantGenusSpecies)

Create empty matrix veg\_matrix \textless- matrix(0, nrow =
length(stands), ncol = length(species), dimnames = list(stands,
species))

And then fill matrix with abundance values using a for loop

for (i in seq\_along(stands)) \{ stand\_data \textless-
subset(veg\_2025, Stand == stands{[}i{]}) for (j in seq\_along(species))
\{ match\_row \textless- stand\_data\(PlantGenusSpecies == species[j]
    if (any(match_row)) {
      veg_matrix[i, j] <- sum(stand_data\)NumPlant{[}match\_row{]},
na.rm = TRUE) \} \} \}

\subsubsection{3) Functions/loops (inputs →
outputs):}\label{functionsloops-inputs-outputs}

I will use the below function to calculate Hill numbers for vegetation
richness, shannon diversity, and inverse simpson index (to give more
weight to common species).

Number of stands (rows) and matrix to fill n\_stands \textless-
nrow(veg\_matrix) richness\_matrix \textless- matrix(0, nrow =
n\_stands, ncol = 3)

Add row and column names rownames(richness\_matrix) \textless-
rownames(veg\_matrix) colnames(richness\_matrix) \textless-
c(``Richness'', ``Shannon'', ``Simpson'')

Using for loop to generate values for (i in 1:n\_stands) \{ stand\_abund
\textless- as.matrix(veg\_matrix{[}i, , drop = FALSE{]})
hill\_matrix{[}i, ``Richness''{]} \textless- hill\_taxa(stand\_abund, q
= 0) hill\_matrix{[}i, ``Shannon''{]} \textless-
hill\_taxa(stand\_abund, q = 1) hill\_matrix{[}i, ``Simpson''{]}
\textless- hill\_taxa(stand\_abund, q = 2) \}

Code for generating Hill dataframe

hill\_df \textless- as.data.frame(hill\_matrix) \%\textgreater\%
rownames\_to\_column(``Stand'')

Code for joining Fire Information with Hill DF

hill\_df \textless- left\_join(hill\_df, fire.div, by = ``Stand'')

Once completed, I will filter for site owner to create new dataframes
that can be compared across owner/site treatments

\subsubsection{4) Statistical test + programmatic
implementation:}\label{statistical-test-programmatic-implementation}

Because I expect Hill numbers to decrease (richness, shannon, and
simpson) as area of high severity fire increase, I will use linear
regression models to display this relationship. Plotted like below,
making sure to account for differing buffers applied to amount of
HighSevFire calculations.

richness\_plots \textless- ggplot(hill\_df, aes(x = HighSevArea, y =
Richness, color = as.factor(Buffer))) + geom\_point(size = 2, alpha =
0.7) + geom\_smooth(method = ``lm'', se = FALSE) + theme\_minimal() +
labs(x = v, y = ``Richness'', color = ``Buffer (m)'') +
facet\_wrap(\textasciitilde{} Buffer, scales = ``free\_x'')

To see if my results are significant agaist null models I will use the
function below for Richness, Shannon, and Simpson; again accounting for
buffer size

lm(formula = Variable \textasciitilde{} HighSeverityArea, data =
filter(richness\_df, Buffer == 500))

\subsubsection{5) Planned
visualizations/tables:}\label{planned-visualizationstables}

For visualization, I plan on using the graphs generated in my linear
regression models. I will also use existing dNBR data from the Holiday
Farm Fire that displays the extent of differing fire severity, overlayed
with Stand survey points. For stand survey points, I will assign a color
based on Hill value.

\subsubsection{6) Risks \& mitigations:}\label{risks-mitigations}

An anticipated risk in my dataset is unbalanced sampling due to
collector mis-id and miscount of plants. The data cleaning and
validating step of my workflow should catch and resolve these errors,
but additional visualizations will highlight any additional outliers.

\subsubsection*{References}\label{references}
\addcontentsline{toc}{subsubsection}{References}

\phantomsection\label{refs}
\begin{CSLReferences}{1}{0}
\bibitem[\citeproctext]{ref-bowdPriorDisturbanceLegacy2021}
Bowd, E. J., Blair, D. P., \& Lindenmayer, D. B. (2021). Prior
disturbance legacy effects on plant recovery post‐high‐severity
wildfire. \emph{Ecosphere}, \emph{12}(5), e03480.
\url{https://doi.org/10.1002/ecs2.3480}

\bibitem[\citeproctext]{ref-cottrellSeedInvasionFilters2008}
Cottrell, T. R., Hessburg, P. F., \& Betz, J. A. (2008). Seed {Invasion}
{Filters} and {Forest} {Fire} {Severity}. \emph{Fire Ecology},
\emph{4}(1), 87--100. \url{https://doi.org/10.4996/fireecology.0401087}

\bibitem[\citeproctext]{ref-fukamiHistoricalContingencyCommunity2015}
Fukami, T. (2015). Historical {Contingency} in {Community} {Assembly}:
{Integrating} {Niches}, {Species} {Pools}, and {Priority} {Effects}.
\emph{Annual Review of Ecology, Evolution, and Systematics},
\emph{46}(1), 1--23.
\url{https://doi.org/10.1146/annurev-ecolsys-110411-160340}

\bibitem[\citeproctext]{ref-halofskyChangingWildfireChanging2020}
Halofsky, J. E., Peterson, D. L., \& Harvey, B. J. (2020). Changing
wildfire, changing forests: The effects of climate change on fire
regimes and vegetation in the {Pacific} {Northwest}, {USA}. \emph{Fire
Ecology}, \emph{16}(1), 4.
\url{https://doi.org/10.1186/s42408-019-0062-8}

\bibitem[\citeproctext]{ref-johnstoneChangingDisturbanceRegimes2016}
Johnstone, J. F., Allen, C. D., Franklin, J. F., Frelich, L. E., Harvey,
B. J., Higuera, P. E., Mack, M. C., Meentemeyer, R. K., Metz, M. R.,
Perry, G. L., Schoennagel, T., \& Turner, M. G. (2016). Changing
disturbance regimes, ecological memory, and forest resilience.
\emph{Frontiers in Ecology and the Environment}, \emph{14}(7), 369--378.
\url{https://doi.org/10.1002/fee.1311}

\bibitem[\citeproctext]{ref-khalifaOverviewBeePollination2021}
Khalifa, S. A. M., Elshafiey, E. H., Shetaia, A. A., El-Wahed, A. A. A.,
Algethami, A. F., Musharraf, S. G., AlAjmi, M. F., Zhao, C., Masry, S.
H. D., Abdel-Daim, M. M., Halabi, M. F., Kai, G., Al Naggar, Y., Bishr,
M., Diab, M. A. M., \& El-Seedi, H. R. (2021). Overview of {Bee}
{Pollination} and {Its} {Economic} {Value} for {Crop} {Production}.
\emph{Insects}, \emph{12}(8), 688.
\url{https://doi.org/10.3390/insects12080688}

\bibitem[\citeproctext]{ref-parksWarmerDrierFire2020}
Parks, S. A., \& Abatzoglou, J. T. (2020). Warmer and {Drier} {Fire}
{Seasons} {Contribute} to {Increases} in {Area} {Burned} at {High}
{Severity} in {Western} {US} {Forests} {From} 1985 to 2017.
\emph{Geophysical Research Letters}, \emph{47}(22), e2020GL089858.
\url{https://doi.org/10.1029/2020GL089858}

\bibitem[\citeproctext]{ref-ponisioPyrodiversityPromotesInteraction2020}
Ponisio, L. C. (2020b). Pyrodiversity promotes interaction
complementarity and population resistance. \emph{Ecology and Evolution},
\emph{10}(10), 4431--4447. \url{https://doi.org/10.1002/ece3.6210}

\bibitem[\citeproctext]{ref-ponisioPyrodiversityPromotesInteraction2020a}
Ponisio, L. C. (2020a). Pyrodiversity promotes interaction
complementarity and population resistance. \emph{Ecology and Evolution},
\emph{10}(10), 4431--4447. \url{https://doi.org/10.1002/ece3.6210}

\bibitem[\citeproctext]{ref-riversReviewResearchNeeds2018}
Rivers, J. W., Galbraith, S. M., Cane, J. H., Schultz, C. B., Ulyshen,
M. D., \& Kormann, U. G. (2018). A {Review} of {Research} {Needs} for
{Pollinators} in {Managed} {Conifer} {Forests}. \emph{Journal of
Forestry}, \emph{116}(6), 563--572.
\url{https://doi.org/10.1093/jofore/fvy052}

\bibitem[\citeproctext]{ref-steelWhenBiggerIsnt2022}
Steel, Z. L., Fogg, A. M., Burnett, R., Roberts, L. J., \& Safford, H.
D. (2022). When bigger isn't better---{Implications} of large
high‐severity wildfire patches for avian diversity and community
composition. \emph{Diversity and Distributions}, \emph{28}(3), 439--453.
\url{https://doi.org/10.1111/ddi.13281}

\bibitem[\citeproctext]{ref-tonEffectsDisturbanceHistory2016}
Ton, M., \& Krawchuk, M. (2016). The {Effects} of {Disturbance}
{History} on {Ground}-{Layer} {Plant} {Community} {Composition} in
{British} {Columbia}. \emph{Forests}, \emph{7}(5), 109.
\url{https://doi.org/10.3390/f7050109}

\bibitem[\citeproctext]{ref-ulyshenForestPollinatorRichness2024}
Ulyshen, M. D., Horn, S., Fair, C., Forrester, E. J., Reynolds, S. K.,
Young, A., \& Schmidt, C. (2024). Forest pollinator richness declines
with distance into burned areas. \emph{Forest Ecology and Management},
\emph{565}, 122049. \url{https://doi.org/10.1016/j.foreco.2024.122049}

\end{CSLReferences}

\end{document}
