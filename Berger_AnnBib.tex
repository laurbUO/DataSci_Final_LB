% Options for packages loaded elsewhere
\PassOptionsToPackage{unicode}{hyperref}
\PassOptionsToPackage{hyphens}{url}
\documentclass[
]{article}
\usepackage{xcolor}
\usepackage[margin=1in]{geometry}
\usepackage{amsmath,amssymb}
\setcounter{secnumdepth}{-\maxdimen} % remove section numbering
\usepackage{iftex}
\ifPDFTeX
  \usepackage[T1]{fontenc}
  \usepackage[utf8]{inputenc}
  \usepackage{textcomp} % provide euro and other symbols
\else % if luatex or xetex
  \usepackage{unicode-math} % this also loads fontspec
  \defaultfontfeatures{Scale=MatchLowercase}
  \defaultfontfeatures[\rmfamily]{Ligatures=TeX,Scale=1}
\fi
\usepackage{lmodern}
\ifPDFTeX\else
  % xetex/luatex font selection
\fi
% Use upquote if available, for straight quotes in verbatim environments
\IfFileExists{upquote.sty}{\usepackage{upquote}}{}
\IfFileExists{microtype.sty}{% use microtype if available
  \usepackage[]{microtype}
  \UseMicrotypeSet[protrusion]{basicmath} % disable protrusion for tt fonts
}{}
\makeatletter
\@ifundefined{KOMAClassName}{% if non-KOMA class
  \IfFileExists{parskip.sty}{%
    \usepackage{parskip}
  }{% else
    \setlength{\parindent}{0pt}
    \setlength{\parskip}{6pt plus 2pt minus 1pt}}
}{% if KOMA class
  \KOMAoptions{parskip=half}}
\makeatother
\usepackage{graphicx}
\makeatletter
\newsavebox\pandoc@box
\newcommand*\pandocbounded[1]{% scales image to fit in text height/width
  \sbox\pandoc@box{#1}%
  \Gscale@div\@tempa{\textheight}{\dimexpr\ht\pandoc@box+\dp\pandoc@box\relax}%
  \Gscale@div\@tempb{\linewidth}{\wd\pandoc@box}%
  \ifdim\@tempb\p@<\@tempa\p@\let\@tempa\@tempb\fi% select the smaller of both
  \ifdim\@tempa\p@<\p@\scalebox{\@tempa}{\usebox\pandoc@box}%
  \else\usebox{\pandoc@box}%
  \fi%
}
% Set default figure placement to htbp
\def\fps@figure{htbp}
\makeatother
% definitions for citeproc citations
\NewDocumentCommand\citeproctext{}{}
\NewDocumentCommand\citeproc{mm}{%
  \begingroup\def\citeproctext{#2}\cite{#1}\endgroup}
\makeatletter
 % allow citations to break across lines
 \let\@cite@ofmt\@firstofone
 % avoid brackets around text for \cite:
 \def\@biblabel#1{}
 \def\@cite#1#2{{#1\if@tempswa , #2\fi}}
\makeatother
\newlength{\cslhangindent}
\setlength{\cslhangindent}{1.5em}
\newlength{\csllabelwidth}
\setlength{\csllabelwidth}{3em}
\newenvironment{CSLReferences}[2] % #1 hanging-indent, #2 entry-spacing
 {\begin{list}{}{%
  \setlength{\itemindent}{0pt}
  \setlength{\leftmargin}{0pt}
  \setlength{\parsep}{0pt}
  % turn on hanging indent if param 1 is 1
  \ifodd #1
   \setlength{\leftmargin}{\cslhangindent}
   \setlength{\itemindent}{-1\cslhangindent}
  \fi
  % set entry spacing
  \setlength{\itemsep}{#2\baselineskip}}}
 {\end{list}}
\usepackage{calc}
\newcommand{\CSLBlock}[1]{\hfill\break\parbox[t]{\linewidth}{\strut\ignorespaces#1\strut}}
\newcommand{\CSLLeftMargin}[1]{\parbox[t]{\csllabelwidth}{\strut#1\strut}}
\newcommand{\CSLRightInline}[1]{\parbox[t]{\linewidth - \csllabelwidth}{\strut#1\strut}}
\newcommand{\CSLIndent}[1]{\hspace{\cslhangindent}#1}
\setlength{\emergencystretch}{3em} % prevent overfull lines
\providecommand{\tightlist}{%
  \setlength{\itemsep}{0pt}\setlength{\parskip}{0pt}}
\usepackage{bookmark}
\IfFileExists{xurl.sty}{\usepackage{xurl}}{} % add URL line breaks if available
\urlstyle{same}
\hypersetup{
  pdftitle={Annotated Bibliography},
  pdfauthor={Lauren Berger},
  hidelinks,
  pdfcreator={LaTeX via pandoc}}

\title{Annotated Bibliography}
\author{Lauren Berger}
\date{}

\begin{document}
\maketitle

\subsubsection{\texorpdfstring{Bowd, E. J. et al.~(2021). \emph{Prior
disturbance legacy effects on plant recovery post‐high-severity
wildfire}}{Bowd, E. J. et al.~(2021). Prior disturbance legacy effects on plant recovery post‐high-severity wildfire}}\label{bowd-e.-j.-et-al.-2021.-prior-disturbance-legacy-effects-on-plant-recovery-posthigh-severity-wildfire}

\textbf{Citation} (Bowd, Blair, and Lindenmayer 2021)

\textbf{Summary}: Explores how past land use and disturbance history
(such as logging and prescribed burning) influence vegetation recovery
after high-severity wildfire events. Using data from long-term
ecological monitoring in southeastern Australia, the authors found that
areas with a legacy of logging or other disturbances exhibited reduced
native plant richness and increased weed invasion following wildfire.
The study emphasizes that pre-fire land management plays a critical role
in shaping post-fire ecological trajectories, with implications for
biodiversity conservation and landscape resilience in fire-prone
regions.

\textbf{Evaluation}: Provides empirical evidence that prior stand
histories impact plant regeneration after fire, serving as a useful
resource in predicting forest recovery. Study is limited in geographic
scope and may not be broadly applicable.

\textbf{Relevance}: Provides a detailed overview on statiscal approaches
for quantifying how varying stand-age catagories influence post fire
species richness.

\subsubsection{\texorpdfstring{Burkle, L. A. et al.~(2015).
\emph{Wildfire disturbance and productivity as drivers of plant species
diversity across spatial
scales}}{Burkle, L. A. et al.~(2015). Wildfire disturbance and productivity as drivers of plant species diversity across spatial scales}}\label{burkle-l.-a.-et-al.-2015.-wildfire-disturbance-and-productivity-as-drivers-of-plant-species-diversity-across-spatial-scales}

\textbf{Citation}(Burkle, Myers, and Belote 2015)

\textbf{Summary}: This study explores the mechanisms by which wildfire
patterns and severity influence biodiversity across environmental
gradients. Found that mixed-diversity fire results in the highest beta
diversity of forb species, and is higher in burned than unburned
landscapes at low productivity sites, but lower at high productivity
sites. Both forbs and woody species had overall higher species richness
in mixed-severity fire sites than high-severity fire sites. Sampled from
plant communities along fire and productivity gradients. Performed
analysis on richness, beta-diversity, and species composition broken
down by plant class between different areas of fire severity.

\textbf{Evaluation}: Strengths: robust sampling and analysis, applicable
to management decisions within a region. Weaknesses: limited in
geographical scope, separates fire into three classes and may leave out
finer scale.

\textbf{Relevance}: Statistical analysis of species richness is
relevant; utilizes Hurlbert's Probability of Interspecific Encounter
(PIE) and a MANOVA. Could be applied, or at least considered, in my
analysis.

\subsubsection{\texorpdfstring{Fukami, T. (2015). \emph{Historical
Contingency in Community Assembly: Integrating Niches, Species Pools,
and Priority
Effects}}{Fukami, T. (2015). Historical Contingency in Community Assembly: Integrating Niches, Species Pools, and Priority Effects}}\label{fukami-t.-2015.-historical-contingency-in-community-assembly-integrating-niches-species-pools-and-priority-effects}

\textbf{Citation}: (Fukami 2015)

\textbf{Summary}: Provides a definition and detailed overview on how
historical contingency and priority effects may impact species pools.
Defines priority effects as the sequence and timing of species arrivals,
caused by historical contingency (past disturbance events). Priority
effects may either preempt or modify niches based on species arrival
time. Explores how regional species pools impacted priority effects come
to be, and the conditions by which historical contingency would be
expected.

\textbf{Evaluation}: This paper serves as a strong conceptual basis and
overview on understanding the factors that influence historical
contingency, and advocates for it's consideration in evaluating
community ecology. Expresses that historical contingency alone cannot
make ecology more predictive, but is an additional tool in organization
framework.

\textbf{Relevanc}e: Provides support for applying a lense of historical
contingency and consideration of priority effects in my analysis on how
plant-pollinator communities are shaped by distance into high severity
fire patches. Emphasizes that species traits and immigration ability are
important considerations in the likeliness of priority effects occuring.
As recolonization is neccesary after a high severity fire events (ie
seed banks and bees wiped out), my analysis will be (an important
evaluation.

\subsubsection{\texorpdfstring{Johnstone, J. F. et al.~(2016).
\emph{Changing disturbance regimes, ecological memory, and forest
resilience}}{Johnstone, J. F. et al.~(2016). Changing disturbance regimes, ecological memory, and forest resilience}}\label{johnstone-j.-f.-et-al.-2016.-changing-disturbance-regimes-ecological-memory-and-forest-resilience}

\textbf{Citation}: (Johnstone et al. 2016)

\textbf{Summary}: This paper provides a comprehensive synthesis
detailing how ecological memory persists in forest ecosystems for years,
decades, or even centuries -- influencing how forests respond to
subsequent disturbances. Past disturbances can influence forest
resilience and trigger reorganizations into new ecosystem states. The
authors highlight that focusing on disturbance legacies can highlight
where forests may be most vulnerable to climate change.

\textbf{Evaluation}: Provides a useful conceptual framework that is
clearly defined in terms of what the impacts of changing disturbance
regimes will be utilizing multiple examples. Not largely quantitative
and limited in geographic scope

\textbf{Relevance}: Provides a useful framework and definitions for
which kinds of legacies may persist in forest following disturbance and
an overview of ecological memory. Serves as a call to test said ideas
presented in the paper to determine how post-disturbance forest dynamics
are playing out in the context of historical contingency.

\subsubsection{\texorpdfstring{Miller, J. E. D., \& Safford, H. D.
(2020). \emph{Are plant community responses to wildfire contingent upon
historical disturbance
regimes?}}{Miller, J. E. D., \& Safford, H. D. (2020). Are plant community responses to wildfire contingent upon historical disturbance regimes?}}\label{miller-j.-e.-d.-safford-h.-d.-2020.-are-plant-community-responses-to-wildfire-contingent-upon-historical-disturbance-regimes}

\textbf{Citation}: (Miller and Safford 2020)

\textbf{Summary}: A large scale meta-analysis focused on studies that
analysed post-fire plant communities. Studies had to have at least two
levels of burn severity to be compared against one and another and
species richness reported as the response variable. 32 studies were
chosen for analysis and all analyzed plot-level (alpha) diversity. While
the authors hypothesize that theory would predict that diversity should
be highest at intermediate disturbance levels, the review finds that
hump-shaped or saturating diversity responses to fire are common in
ecosystems with historically low- to moderate-severity fires. However,
in ecosystems historically dominated by high-severity fires, diversity
often peaks after high-severity disturbances. These patterns contradict
traditional theory, which predicts stronger negative effects of
disturbance in low-productivity systems. The authors suggest that
historical fire regimes (i.e.~historical contingency) should be included
in disturbance-diversity theory alongside disturbance frequency,
intensity, and productivity.

\textbf{Evaluation}: Because this research is a meta-analysis - it does
not delve into community response, however the studies cited stand as a
useful starting point fo similiar study designs. It is alo difficult to
compare studies that use different field studies and there is a
relatively limited body of research on multiple fire regimes that is
lacking in geographic exctent.

\textbf{Relevance}: This analysis provides support for applying
historical contingency in consideration of fires impact on diversity.
Given the cascades history of fire, simply considering disturbance
theory may not be sufficient.

\subsubsection{\texorpdfstring{Ponisio, L. C. et al.~(2016).
\emph{Pyrodiversity begets plant-pollinator community
diversity}}{Ponisio, L. C. et al.~(2016). Pyrodiversity begets plant-pollinator community diversity}}\label{ponisio-l.-c.-et-al.-2016.-pyrodiversity-begets-plant-pollinator-community-diversity}

\textbf{Citation}: (Ponisio et al. 2016)

\textbf{Summary}: Explores whether pyrodiversity impacts biodiversity.
The study focused on the llilouette Creek Basin of Yosemite National
Park, in the central Sierra Nevada of California. Plant and pollinator
community surveys were conducted 9- 12 years after fire and surveyed a
gradient of fire diversity as confirmed by dNBR information. Generalized
linear mixed models were utilized to determine if plant-pollinator
richness for each sample site was impacted by pyrodiversity. The same
testing was conducted for environmental heterogeneity to analyze the
additional impact of drought, and species turnover between sites was
also analyzed. Results indicate that higher pyrodiversity was correlated
with higher richness of flowering plants, pollinators, and
plant--pollinator interactions. \textbf{Evaluation}: This study captures
a large area, pyrodiversity gradient, and conducted a clear interaction
analysis. The limitations of this study in evaluating pyrodiversity are
that studies were conducted during a drought year and may be limited in
applicability to pyrodiversity analysis on sites unimpacted by drought.

\textbf{Relevance}: I will draw primarily on the richness comparison and
incorporate similar methods in my analysis. This paper also provides a
relevant and useful framework for calculating pyro-divesity utilizing
dNBR information.

\subsubsection{\texorpdfstring{Reilly, M. J. (Year). \emph{Cascadia
Burning: The historic, but not historically unprecedented, 2020
wildfires in the Pacific Northwest,
USA}}{Reilly, M. J. (Year). Cascadia Burning: The historic, but not historically unprecedented, 2020 wildfires in the Pacific Northwest, USA}}\label{reilly-m.-j.-year.-cascadia-burning-the-historic-but-not-historically-unprecedented-2020-wildfires-in-the-pacific-northwest-usa}

\textbf{Citation}: (Reilly et al. 2022)

\textbf{Summary}: This paper provides a comprehensive synthesis of the
environmental drivers that influenced the 2020 labor day fires in the
West Cascades. While these fires were labeled as an unprecedented
historical event in terms of size and severity for the region, reports
from the early 20th century and paleo- and dendro-ecological records
indicate that similar and potentially even larger wildfires have
occurred over the past millennium -- sharing many of the same
characteristics. This paper explores the ecological impacts of large
fire events in the Western Cascades -- stating that high-severity fire
events can generate early seral habitat that foster biodiversity -- but
can also lead to habitat fragmentation and the introduction of
non-native species that can have detrimental effects on post-fire plant
and pollinator communities.

\textbf{Evaluation}: This paper provides a detailed and well cited
history of historical fire regimes into analyzing if the 2020 labor day
fires were a result of climate change, or more consistent with
historical fire. A potential weakness that the authors state is that the
impacts of climate change on dry, eastside winds that caused fire are
still poorly understood. Another potential weakness is that this paper
evaluates early colonial records, but has limited input of indigenous
records pre-dating colonialism about fire history. While Paleoecological
records reflect a history of fire in the region, it does not draw on how
impacts of logging practices post-colonialism may have influenced the
2020 labor day fires.

\textbf{Relevance}: This paper provides a reframing of how to view fire
history in the region, providing empirical evidence that the 2020 labor
day fires are not entirely something new for this region. Also raises
important considerations for variables that are likely to influence
landscape following fire events, such as species.

\subsubsection{\texorpdfstring{Steel, Z. L. et al.~(2022). \emph{When
bigger isn't better---Implications of large high-severity wildfire
patches for avian diversity and community
composition}}{Steel, Z. L. et al.~(2022). When bigger isn't better---Implications of large high-severity wildfire patches for avian diversity and community composition}}\label{steel-z.-l.-et-al.-2022.-when-bigger-isnt-betterimplications-of-large-high-severity-wildfire-patches-for-avian-diversity-and-community-composition}

\textbf{Citation}(Steel et al. 2022)

\textbf{Summary}: Evaluates the impact of increasing high-severity fire
patch size, and increased distance from unimpacted edges, on bird
communities. Utilizes multi-year avian-point count surveys completed in
27 different fires; collected 10 or more years since disturbance and
consisting of varying levels of patch sizes. Analysis of both alpha and
beta diversity found that in comparison to patch edges, and smaller
patches of high-severity fire, interiors of large patches supported
fewer species and displayed lower rates of species richness since the
time of last fire.

\textbf{Evaluation}: As with any ecological field study, sampling
methods may favor certain bird species. While a null model is not
explicitly addressed, the authors cite (Chase et al., 2011) and utilize
a Raup--Crick Index in their analysis of beta-diversity to adjust for
differences in species richness. In their alpha diversity analysis,
assessment is based on only species richness, as an analysis of species
abundance cannot be completed with only occupancy models.

\textbf{Relevance}: This paper is very relevant to my research, as I
will be completing a similar analysis, but with plants and pollinators.
The statistical methods employed to calculate alpha diversity and amount
of high severity fire surrounding a site will both inform my analysis
and how I choose to model my results. The code for this paper is shared
in Github.

\subsubsection{\texorpdfstring{Ton, M., \& Krawchuk, M. A. (2016).
\emph{The Effects of Disturbance History on Ground-Layer Plant Community
Composition in British
Columbia}}{Ton, M., \& Krawchuk, M. A. (2016). The Effects of Disturbance History on Ground-Layer Plant Community Composition in British Columbia}}\label{ton-m.-krawchuk-m.-a.-2016.-the-effects-of-disturbance-history-on-ground-layer-plant-community-composition-in-british-columbia}

\textbf{Citation}(Ton and Krawchuk 2016)

\textbf{Summary}: Evaluates if the double disturbance of logging
followed in quick succession by high-severity fire impacts plant
community composition in comparison to sites only impacted by high
severity fire and left unlogged. Plant community data was collected from
31 sites, and all vascular plants and bryophytes were identified to
species when possible. Plant traits of height, length, and leaf area
data were also recorded. Analysis of community richness, diversity, and
evenness was completed, in addition to analysis on traits, indicator
species, and composition. Results found a significant difference in
taxonomic composition and shrub mass between sites, but not in species
richness, diversity, and evenness.

\textbf{Evaluation}: This study is able to utilize a natural experiment
design to compare to treatments (logged and unlogged). The field sample
is robust, and represents all plant groups. This study is limited in
temporal scale, and produced modest results that may be lacking in
consideration of broader environmental influences (proximity to roads
and unburned locations).

\textbf{Relevance}: As my analysis will be examining similar questions,
this paper supports the utilization of the Shannon-Wiener index to
calculate diversity. This paper also presents an approach I had not
considered of using Canonical Correspondence Analysis (CCA) which allows
for accounting of environmental variability in soil pH and canopy cover.
While I don't have detailed information on these environmental
conditions, I may be able to incorporate it into my analysis.

\subsubsection{\texorpdfstring{Ulyshen, M. D. et al.~(Year).
\emph{Forest pollinator richness declines with distance into burned
areas}}{Ulyshen, M. D. et al.~(Year). Forest pollinator richness declines with distance into burned areas}}\label{ulyshen-m.-d.-et-al.-year.-forest-pollinator-richness-declines-with-distance-into-burned-areas}

\textbf{Citation}: (Ulyshen et al. 2024)

\textbf{Summary}: Evaluates how pollinator richness and abundance are
impacted by increased distance into burned areas. Researchers conducted
pollinator surveys from edges of burned areas towards the center along
500 meters. Richness and abundance were calculated as response variables
based upon distance into these burned areas. The authors found a
significant decline in richness at the center of burn units, but
abundance did not show a decline. However, in accounting for pollinator
size, the abundance of small pollinators decreased and large pollinators
increased with distance into a burn unit. Multiple models were utilized
in R to test which most accurately modeled these abundance and richness
relationships.

\textbf{Evaluation}: The strengths of this study are that a it covers a
broad taxonomic range of pollinators -- not just bees -- to analysis
impact on pollinator communities as a whole. The study design is
centered on sites with similar treatments (all in controlled burns),
providing a homogenous burn area. This is both a strength, and weakness,
as the proximity to unburned habitats may not be reflective wildfire. As
with all pollinator surveys, possible bias may exist in sampling.

\textbf{Relevance}: Serves as a useful comparison for testing both
richness and abundance in proximity for distance into burn. Their
statistical analysis portion offers a summary of several different
models, and posses an alternative of not only looking at abundance but
abundance based on size. This research also provides support for the
analysis I want to conduct, and suggests a strong relationship between
proximity to unburned edges and pollinator richness.

\phantomsection\label{refs}
\begin{CSLReferences}{1}{0}
\bibitem[\citeproctext]{ref-bowdPriorDisturbanceLegacy2021}
Bowd, Elle J., David P. Blair, and David B. Lindenmayer. 2021. {``Prior
Disturbance Legacy Effects on Plant Recovery Post‐high‐severity
Wildfire.''} \emph{Ecosphere} 12 (5): e03480.
\url{https://doi.org/10.1002/ecs2.3480}.

\bibitem[\citeproctext]{ref-burkleWildfireDisturbanceProductivity2015}
Burkle, Laura A., Jonathan A. Myers, and R. Travis Belote. 2015.
{``Wildfire Disturbance and Productivity as Drivers of Plant Species
Diversity Across Spatial Scales.''} \emph{Ecosphere} 6 (10): 1--14.
\url{https://doi.org/10.1890/ES15-00438.1}.

\bibitem[\citeproctext]{ref-fukamiHistoricalContingencyCommunity2015}
Fukami, Tadashi. 2015. {``Historical {Contingency} in {Community}
{Assembly}: {Integrating} {Niches}, {Species} {Pools}, and {Priority}
{Effects}.''} \emph{Annual Review of Ecology, Evolution, and
Systematics} 46 (1): 1--23.
\url{https://doi.org/10.1146/annurev-ecolsys-110411-160340}.

\bibitem[\citeproctext]{ref-johnstoneChangingDisturbanceRegimes2016}
Johnstone, Jill F, Craig D Allen, Jerry F Franklin, Lee E Frelich, Brian
J Harvey, Philip E Higuera, Michelle C Mack, et al. 2016. {``Changing
Disturbance Regimes, Ecological Memory, and Forest Resilience.''}
\emph{Frontiers in Ecology and the Environment} 14 (7): 369--78.
\url{https://doi.org/10.1002/fee.1311}.

\bibitem[\citeproctext]{ref-millerArePlantCommunity2020}
Miller, Jesse E. D., and Hugh D. Safford. 2020. {``Are Plant Community
Responses to Wildfire Contingent Upon Historical Disturbance Regimes?''}
Edited by Benjamin Poulter. \emph{Global Ecology and Biogeography} 29
(10): 1621--33. \url{https://doi.org/10.1111/geb.13115}.

\bibitem[\citeproctext]{ref-ponisioPyrodiversityBegetsPlant2016}
Ponisio, Lauren C., Kate Wilkin, Leithen K. M'Gonigle, Kelly Kulhanek,
Lindsay Cook, Robbin Thorp, Terry Griswold, and Claire Kremen. 2016.
{``Pyrodiversity Begets Plant--Pollinator Community Diversity.''}
\emph{Global Change Biology} 22 (5): 1794--1808.
\url{https://doi.org/10.1111/gcb.13236}.

\bibitem[\citeproctext]{ref-reillyCascadiaBurningHistoric2022}
Reilly, Matthew J., Aaron Zuspan, Joshua S. Halofsky, Crystal Raymond,
Andy McEvoy, Alex W. Dye, Daniel C. Donato, et al. 2022. {``Cascadia
{Burning}: {The} Historic, but Not Historically Unprecedented, 2020
Wildfires in the {Pacific} {Northwest}, {\textless{}}Span
Style="font-Variant:small-Caps;"{\textgreater{}}{USA}{\textless{}}/Span{\textgreater{}}.''}
\emph{Ecosphere} 13 (6): e4070. \url{https://doi.org/10.1002/ecs2.4070}.

\bibitem[\citeproctext]{ref-steelWhenBiggerIsnt2022}
Steel, Zachary L., Alissa M. Fogg, Ryan Burnett, L. Jay Roberts, and
Hugh D. Safford. 2022. {``When Bigger Isn't Better---{Implications} of
Large High‐severity Wildfire Patches for Avian Diversity and Community
Composition.''} Edited by Sally Archibald. \emph{Diversity and
Distributions} 28 (3): 439--53. \url{https://doi.org/10.1111/ddi.13281}.

\bibitem[\citeproctext]{ref-tonEffectsDisturbanceHistory2016}
Ton, Michael, and Meg Krawchuk. 2016. {``The {Effects} of {Disturbance}
{History} on {Ground}-{Layer} {Plant} {Community} {Composition} in
{British} {Columbia}.''} \emph{Forests} 7 (5): 109.
\url{https://doi.org/10.3390/f7050109}.

\bibitem[\citeproctext]{ref-ulyshenForestPollinatorRichness2024}
Ulyshen, Michael D., Scott Horn, Conor Fair, Emily J. Forrester, Samm K.
Reynolds, Andrew Young, and Carl Schmidt. 2024. {``Forest Pollinator
Richness Declines with Distance into Burned Areas.''} \emph{Forest
Ecology and Management} 565 (August): 122049.
\url{https://doi.org/10.1016/j.foreco.2024.122049}.

\end{CSLReferences}

\end{document}
